\documentclass{beamer}

\usepackage[utf8]{inputenc}
\usepackage[T1]{fontenc}
\usepackage[english]{babel}
\usepackage{tikz}
\usepackage{algorithm}
\usepackage{algorithmicx}
\usepackage{algpseudocode}
\usepackage{amsmath}
%\usepackage{pgfplots}
%\usepackage{svg}

\usetheme{Madrid}

\title{Dynamical Processes : Presentation}
\subtitle{Anomalous versus Slowed-Down Brownian Diffusion
in the Ligand-Binding Equilibrium \\ H. Soula et. al.}
\author{Marc Heinrich \\ Clément Henin}
\date\today

\AtBeginSection[]{
  \stepcounter{subsection}
  \begin{frame}
  \vfill
  \centering
  \begin{beamercolorbox}[sep=8pt,center,shadow=true,rounded=true]{title}
    \usebeamerfont{title}\insertsectionhead\par%
  \end{beamercolorbox}
  \vfill
  \end{frame}
}

\begin{document}

\begin{frame}
\maketitle
\end{frame}

\begin{frame}
\frametitle{Introduction}
%Justification of the study: 
	% Anomalous diffusion in molecules
\end{frame}

\section{Two Models}

\begin{frame}{Brownian motion with obstacles}
% Put the definition here
% General behaviour : brownian motion at small and large time scales
\end{frame}

\begin{frame}
%Put the short movie here
\end{frame}

\begin{frame}{CTRW}
% Definition + general behaviour (mean square deviation)
\end{frame}

\begin{frame}
% Problem studied : Ligand binding equation
% Put the equation
% Say a word why it is important
\end{frame}

\begin{frame}
%Time evolution of number of C for the brownian model
% For several values of D
%Value at equilibrium does not depend on D
\end{frame}

\begin{frame}
% Same with several values of rho
% Modification in the value at equilibrium
\end{frame}

\section{Simulations}

\begin{frame}
%Graph of equilibrium value as a function of $D$
\end{frame}

\begin{frame}
%Graph of equilibrium value as a function of rho
%Good fit for low values rho
% Less good when we get close to percolation threshold
\end{frame}

\begin{frame}

\end{frame}


\end{document}